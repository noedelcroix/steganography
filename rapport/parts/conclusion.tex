\newpage
\section{Conclusion}
En conclusion, nous avons vu deux techniques de stéganographies parmies une innombrable liste de techniques.
\newline
La technique LSB permet donc de cacher des données dans une image au format PNG et est sans doute compatible avec n'importe quel format d'image codant ses pixels en RVB. Cette technique limite la taille maximale des données à cacher au nombre de pixels de l'image fois 3/8 (3 canaux de couleurs et 8 bits par caractère unicode) en octets.
\newline
La technique zero-width est un peu plus exotique et permet de cacher des données dans un fichier texte brut. Cette technique de stéganographie ne limite pas la taille des données à insérer mais celle-ci augmente considèrablement la taille du fichier, là où la technique LSB n'allourdit pas le fichier.