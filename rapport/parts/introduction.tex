\newpage
\section{Introduction}
Dans ce rapport, je vais vous présenter le résultat de mon travail de recherche au sujet de la stéganographie et plus précisément les techniques less significant bits (LSB) et zero-width characters.
\newline
La stéganographie est un ensemble de techniques permettant de cacher des données au sein d'un autre fichier. La stéganographie n'a rien de nouveau, peut s'appliquer sur plusieurs supports que ce soit des documents physiques (papier, pierre, photos, etc.) ou informatisé (images numériques, fichiers textes, etc.).
\newline
dans ce rapport, je vais présenter deux techniques. L'une permettant de stocker des données dans les pixels que compose une image PNG (ou tout autre format d'image n'utilisant pas de compression et représentant ces pixels en true color/RVB), l'autre permet de stocker n'importe quelle donnée dans un fichier texte ou tout autre format ne vérifiant pas l'intégrité de ses données. 